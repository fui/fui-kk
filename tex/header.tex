\documentclass[norsk,a4paper,11pt]{report}
\usepackage[T1]{fontenc}
\usepackage[utf8]{inputenc}
\usepackage{helvet}
\usepackage{babel,epsfig,moreverb,ifikompendiumforside,float}

\author{Fagutvalget ved Institutt for informatikk}
% m fylles ut, f.eks INxxx-x
\title{Kursevaluering}
% f.eks `Obligatorisk oppg. x'
\subtitle{Høst 2015}

\date{Mai 2016}

\floatstyle{ruled}
\restylefloat{figure}

% for paragrafer uten innrykk og tomme linjer mellom dem:
\setlength{\parindent}{0pt}
\setlength{\parskip}{1ex plus 0.5ex minus 0.2ex}

% Lagt inn av josek 19. okt 2009, for bedre utnyttelse av papirarealet:
\setlength{\textwidth}		{15.0cm}
\setlength{\textheight}		{24.5cm}

\begin{document}

\ififorside{}

% Lagt inn av josek 19. okt 2009, disse er best å ha etter uiosloforside:
\setlength{\topmargin}		{-1.0cm}
\setlength{\headsep}		{0.5cm}
\setlength{\oddsidemargin}	{0.5cm}
\setlength{\evensidemargin}	{0.5cm}
\setlength{\footskip}		{1.0cm}

% --- eps figur ---
%
%\begin{figure}[htbp!]
%  \caption{En fin liten figur}
%  \label{fig:minfigur}
%  \begin{center}
%    \epsfig{file=minfigur.eps,angle=90,width=\textwidth}
%  \end{center}
%\end{figure}

% --- textfil med tab ---
%
%\verbatimtabinput{textfil.txt}
