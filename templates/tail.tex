\chapter{Avslutning}
\label{chp:end}
\newpage

\section{Om Kursevaluering}
Kursevalueringen gjennomføres av frivillige studenter i Fagutvalget ved Institutt for Informatikk.
Semesterets kursevaluering er gjennomført av:\\

\textbf{Undervisningsansvarlig:} Person X\\

\textbf{Kursevalueringsgruppa:}
\begin{itemize}[label={}]
\item Person 1
\item Person 2
\item Person 3
\item Person 4
\item Person 5
\item Person 6
\item Person 7
\item Person 8
\end{itemize}

Takk til alle involverte, øvrige medlemmer i FUI og IFI!

\section{Teknisk løsning - Open Source}
\label{sec:OS}
Den tekniske løsningen rundt kursevalueringen er et åpen-kildekode prosjekt, tilgjengelig på GitHub:
\begin{center}
\href{https://github.com/fui/fui-kk}{\texttt{github.com/fui/fui-kk}}
\end{center}

Kursevalueringen bruker nettskjema, utviklet av USIT, for å samle inn svar anonymt fra studenter oppmeldt til fag på IFI. FUIs egenutviklede Python programvare, \href{https://github.com/fui/fui-kk}{\texttt{fui-kk}}, laster ned regneark og rapporter fra nettskjema, behandler disse og regner ut statistikk for flervalgsspørsmål. Dataene blir kombinert med sammendrag skrevet av frivillige i FUI og alt settes sammen til en stor \LaTeX \space rapport som leveres til Undervisningsutvalget(UU). Fullstendige rapporter (rådata) for enkeltfag blir også gjort tilgjengelig for faglærere og oppsummerte rapporter publiseres for studenter og andre UiO-ansatte på innlogget side.
\newpage
\section{Fagutvalget ved Institutt for Informatikk}
Fagutvalget ved Institutt for informatikk (FUI) er informatikkstudentenes eget organ, og skal
fungere som et bindeledd mellom studentene, instituttet og universitetet forøvrig. FUI har mellom 7 og 17 medlemmer, valgt på et allmøte for instituttets studenter i begynnelsen av hvert semester.

Mer informasjon om FUI finnes på nett:
\begin{center}
\href{http://ififui.no/}{\texttt{ififui.no}}
\end{center}

\begin{figure}[H]
\begin{center}
\includegraphics[width=0.3\textwidth]{../../../../resources/fui/fui-logo.png}
\end{center}
\end{figure}

\end{document}
