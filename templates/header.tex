\documentclass[norsk,a4paper,11pt]{report}
\usepackage[T1]{fontenc}
\usepackage[utf8]{inputenc}
\usepackage{helvet}
\usepackage{babel,epsfig,moreverb,ifikompendiumforside,float}
\usepackage{enumitem}
\usepackage{color}
\usepackage{etoolbox}
\usepackage{varioref}
\usepackage[
    colorlinks,
    citecolor=black,              % I like links with standard black color
    filecolor=black,
    linkcolor=black,
    urlcolor=black
]{hyperref}                       % Links in TOC etc.
\usepackage[all]{hypcap}          % Better links to floating environment

\author{Fagutvalget ved Institutt for informatikk}
\title{Kursevaluering}
\subtitle{Vår 2016}

\date{\today}

\AtBeginEnvironment{quote}{\itshape}

\floatstyle{ruled}
\restylefloat{figure}

% for paragrafer uten innrykk og tomme linjer mellom dem:
\setlength{\parindent}{0pt}
\setlength{\parskip}{1ex plus 0.5ex minus 0.2ex}

% Lagt inn av josek 19. okt 2009, for bedre utnyttelse av papirarealet:
\setlength{\textwidth}      {15.0cm}
\setlength{\textheight}     {24.5cm}

\begin{document}

\ififorside{}

% Lagt inn av josek 19. okt 2009, disse er best å ha etter uiosloforside:
\setlength{\topmargin}      {-1.0cm}
\setlength{\headsep}        {0.5cm}
\setlength{\oddsidemargin}  {0.5cm}
\setlength{\evensidemargin} {0.5cm}
\setlength{\footskip}       {1.0cm}
\renewcommand{\thesection}{}
\renewcommand{\thesubsection}{}
\tableofcontents
\newpage
\chapter{Introduksjon}
\label{chp:intro}
\section{FUIs kursevaluering}
Hvert semester utfører Fagutvalget ved Institutt for Informatikk (FUI) en evaluering av alle bachelor- og master-emner på Institutt for Informatikk (IFI).

I 2016 introduserte FUI en ny teknisk løsning rundt kursevalueringen.
Generell vurdering som vises på grafer er regnet ut basert på et enkelt gjennomsnitt, også tilbake i tid.
Dette medfører en finere oppdelt skala og vi kan se trender bedre.
For mer informasjon se \textbf{\hyperref[sec:OS]{Kapittel 3: Teknisk løsning - Open source}} i slutten av rapporten.

\section{Deltakerstatistikk}
Nedenfor vises deltakelse i FUIs kursevaluering fordelt på kurs, for alle kurs med 100 eller flere inviterte studenter:
\input{../../outputs/participation.tex}

\section{Fag FUI mener bør trekkes frem}
Hvert år presenterer FUI noen fag som gjør det veldig bra og noen som har forbedringspotensiale.
FUI ser på det som svært positivt at mange fag får meget gode tilbakemeldinger og at det derfor er vanskelig å nevne noen, men utelate andre.

\hyperref[course:INF1234]{\textbf{INF1234 - Eksempelkurs}} har siden V2012 gått fra vurderingen  \emph{Bra} til \emph{Meget bra}.

\begin{quote}
I felt exemplified.
\end{quote}

\section{Å lese kursevalueringen}
Hvert kurs er tredelt og består av
\begin{itemize}
    \item Statistikk og graf, generert basert på flervalgsspørsmål.
    \item Oppsummering av studentenes meninger, skrevet av FUI-medlemmer.
    \item Utvalgte representative sitater.
\end{itemize}
Kurs med veldig få respondenter er med for å kunne se statistikken, og får kortere oppsummeringer og færre sitater.

\newpage
\chapter{Kurs}
\label{chp:courses}
\newpage
